% Oliver Kullmann, 23.12.1999 (Toronto)
% Copyright 1999 - 2010 Oliver Kullmann
% This file is part of the OKlibrary. OKlibrary is free software; you can redistribute 
% it and/or modify it under the terms of the GNU General Public License as published by
% the Free Software Foundation and included in this library; either version 3 of the 
% License, or any later version.
%
% Die Basis fuer Folien wie fuer normale Texte.
%
% -----------------------------------------------------------------------------------------------------------------
% ENVIRONMENT
% -----------------------------------------------------------------------------------------------------------------
%
\scrollmode
%
\usepackage{amsmath}
\usepackage{amsfonts}
\usepackage{amssymb}
\usepackage{latexsym}
\usepackage{stmaryrd}
\usepackage{array}
\usepackage{exscale}
%
\renewcommand{\baselinestretch}{1.05}
%
%
\newcommand{\BibliographyOKlibrary}{\bibliography{%
  Latex_bib/MiscSAT,%
  Latex_bib/Constraints,%
  Latex_bib/C++,%
  Latex_bib/AlgorithmsDataStructures,%
  Latex_bib/GeneralProgramming%
}
}
% Adressen (18.4.2004)
\newcommand{\AdresseSwansea}{\author{Oliver Kullmann\\[1ex]
  Computer Science Department\\
  University of Wales Swansea\\
  Swansea, SA2 8PP, UK\\[1ex]
 e-mail: O.Kullmann@Swansea.ac.uk\\
  {\small
    http://cs-svr1.swan.ac.uk/\rilde{}csoliver/}}
}
%
\newcommand{\AdresseSwanseaEPSRCOne}{\author{Oliver Kullmann\thanks{Supported by EPSRC Grant GR/S58393/01}\\[1ex]
  Computer Science Department\\
  University of Wales Swansea\\
  Swansea, SA2 8PP, UK\\[1ex]
 e-mail: O.Kullmann@Swansea.ac.uk\\
  {\small
    http://cs-svr1.swan.ac.uk/\rilde{}csoliver/}}
}
\newcommand{\AdresseSwanseaOKMH}{\author{Oliver Kullmann\thanks{Supported by EPSRC Grant GR/S58393/01}\\
  Computer Science Department\\
  University of Wales Swansea\\
  Swansea, SA2 8PP, UK\\
 e-mail: O.Kullmann@Swansea.ac.uk\\
  {\small
    http://cs-svr1.swan.ac.uk/\rilde{}csoliver/}
\and Matthew Henderson\thanks{Supported by EPSRC Grant GR/S58393/01}\\
  Computer Science Department\\
  University of Wales Swansea\\
  Swansea, SA2 8PP, UK\\
 e-mail: M.Henderson@Swansea.ac.uk\\
  {\small
    http://XXX}}
}
\newcommand{\AdresseSwanseaMHOK}{\author{Matthew Henderson\thanks{Supported by EPSRC Grant GR/S58393/01}\\
  Computer Science Department\\
  University of Wales Swansea\\
  Swansea, SA2 8PP, UK\\
 e-mail: M.Henderson@Swansea.ac.uk\\
  {\small
    http://cs-svr1.swan.ac.uk/\rilde{}csoliver/}
\and Oliver Kullmann\thanks{Supported by EPSRC Grant GR/S58393/01}\\
  Computer Science Department\\
  University of Wales Swansea\\
  Swansea, SA2 8PP, UK\\
 e-mail: O.Kullmann@Swansea.ac.uk\\
  {\small
    http://cs-svr1.swan.ac.uk/\rilde{}csoliver/}}
}
\newcommand{\AdresseSwanseaOKILJMS}{\author{Oliver Kullmann\thanks{Supported by EPSRC Grant GR/S58393/01}\\
  Computer Science Department\\
  University of Wales Swansea\\
  Swansea, SA2 8PP, UK\\
 e-mail: O.Kullmann@Swansea.ac.uk\\
  {\small
    http://cs-svr1.swan.ac.uk/\rilde{}csoliver/}
  \and In{\^{e}}s Lynce\\
  Departamento de Engenharia Informatica\\
  Instituto Superior Tecnico\\
  Universidade Tecnica de Lisboa\\
  e-mail: ines@sat.inesc-id.pt\\
  {\small
    http://sat.inesc-id.pt/\rilde{}ines}
  \and Jo{\~{a}}o Marques-Silva\\
  School of Electronics and Computer Science\\
  University of Southampton\\
  Highfield, Southampton SO17 1BJ, UK\\
  e-mail: jpms@soton.ac.uk\\
  {\small
    http://www.ecs.soton.ac.uk/\rilde{}jpms/}}
}

% ########################################################
% ----------------------------------------------------------------------------------------------------------------
% GENERAL CONSTRUCTS
% -----------------------------------------------------------------------------------------------------------------
% ########################################################
\newcommand{\nc}{\newcommand}
\newcommand{\ol}{\overline}
\newcommand{\ul}{\underline}
\newcommand{\es}{\emptyset}
\newcommand{\sm}{\setminus}
\newcommand{\ve}{\varepsilon}
\newcommand{\vp}{\varphi}
\newcommand{\bw}{\bigwedge}
\newcommand{\bv}{\bigvee}
\newcommand{\bc}{\bigcup}
\newcommand{\bca}{\bigcap}
\newcommand{\Lra}{\Leftrightarrow}
\newcommand{\Lora}{\Longrightarrow}
\newcommand{\Lla}{\Longleftarrow}
\newcommand{\Llra}{\Longleftrightarrow}
\newcommand{\Ra}{\Rightarrow}
\newcommand{\La}{\Leftarrow}
\newcommand{\ra}{\rightarrow}
\newcommand{\lora}{\longrightarrow}
\newcommand{\la}{\leftarrow}
\newcommand{\lra}{\leftrightarrow}
\newcommand{\da}{\downarrow}
\newcommand{\ub}{\underbrace}
\newcommand{\ob}{\overbrace}
\newcommand{\sst}{\subset}
\newcommand{\sse}{\subseteq}
\newcommand{\spt}{\supset}
\newcommand{\spe}{\supseteq}
\newcommand{\fa}{\forall}
\newcommand{\ex}{\exists}
\newcommand{\mr}{\mathrm}
\newcommand{\mc}{\mathcal}
\newcommand{\mf}{\mathfrak}
\newcommand{\vtr}{\vartriangleright}
\newcommand{\trd}{\triangledown}
\newcommand{\DMO}{\DeclareMathOperator}
%
\newcommand{\DST}{\displaystyle}
\newcommand{\TST}{\textstyle}
\newcommand{\SST}{\scriptstyle}
\newcommand{\SSST}{\scriptscriptstyle}
%
\newcommand{\ZZ}{\mathbb{Z}}
\newcommand{\NN}{\mathbb{N}}
\newcommand{\NNZ}{\NN_0}
\newcommand{\QQ}{\mathbb{Q}}
\newcommand{\RR}{\mathbb{R}}
\newcommand{\CC}{\mathbb{C}}
\newcommand{\PP}{\mathbb{P}}
\newcommand{\AAM}{\mathbb{A}}
\newcommand{\MM}{\mathbb{M}}
\newcommand{\GG}{\mathbb{G}}
\newcommand{\BB}{\mathbb{B}}
\newcommand{\KK}{\mathbb{K}}
\newcommand{\EE}{\mathbb{E}}
\newcommand{\FF}{\mathbb{F}}
\newcommand{\WW}{\mathbb{W}}
\newcommand{\LL}{\mathbb{L}}
\newcommand{\HH}{\mathbb{H}}
\newcommand{\UU}{\mathbb{U}}
\newcommand{\OO}{\mathbb{O}}
\newcommand{\SSM}{\mathbb{S}}
%
\newcommand{\Ende}{\ \rule{0.4em}{1.7ex}}
\newcommand{\pr}{\noindent\textbf{Proof:}\quad}
\newcommand{\prd}{\noindent\textbf{Beweis:}\quad}
%%
\newcommand{\ha}{\hspace*{5mm}}
\newcommand{\hb}{\hspace*{10mm}}
\newcommand{\hc}{\hspace*{15mm}}
\newcommand{\hd}{\hspace*{20mm}}
\newcommand{\he}{\hspace*{25mm}}
%
\newcommand{\mar}[1]{\makebox[0cm]{#1}}
%
% 16.10.2003
\newcommand{\NAA}{\setlength{\itemsep}{0pt}} % Null Aufzaehlungs-Abstand
%
% Tilde-Symbol ~ ("richtig"); fragiles Kommando (wg. \raisebox)
% Do not use within the \url command --- there just use "~" !
\newcommand{\rilde}{{\large\raisebox{-1ex}[0mm][0mm]{\~{}}\hspace{-0.05em}}}
%
% Quotierung (30.11.2003)
\newcommand{\Quot}[1]{\glqq{}#1\grqq{}}
\newcommand{\Qu}[1]{\Quot{#1}}
%
% Fuer xymatrix (18.11.2003)
\newcommand{\aru}{\ar @{-}} % ungerichtete Kante
\newcommand{\arug}{\ar @{--}} % gestrichelte ungerichtete Kante
\newcommand{\ardg}{\ar @{-->}} % gestrichelte gerichtete Kante
%
% Wiedergabe von Programmen
\usepackage{listings}
\lstloadlanguages{Pascal,C++}
\newcommand{\Pascal}{\lstset{language=Pascal,keywordstyle=\bfseries,breaklines,breakindent=30pt}}
\newcommand{\Cpp}{\lstset{language=C++,keywordstyle=\bfseries,breaklines,breakindent=30pt}}
\newcommand{\inl}[1]{\lstinline$#1$}
% ########################################################
% ----------------------------------------------------------------------------------------------------------------
% MATHEMATICAL NOTATIONS
% -----------------------------------------------------------------------------------------------------------------
% ########################################################
\newcommand{\und}{{\:\wedge\:}}
\newcommand{\oder}{{\:\vee\:}}
%
\newcommand{\mb}{{\:|\:}} % Mengenbildner
\newcommand{\set}[1]{\{ #1 \}}
\newcommand{\setb}[1]{\big \{ \, #1 \, \big \}}
%
\newcommand{\rbca}[1]{\bca_{#1}\nolimits} % relative intersection
% ToDo: How to produce \rbca{X}_{i \in I} ?! (\sideset is too restricted, since the symbol
% will always be in displaystyle).
%
\DMO{\dom}{dom}
\DMO{\id}{id}
\DMO{\cod}{cod} % Codomain (28.11.2003)
\DMO{\rg}{rg} % Wertemenge ("range")
%
\DMO{\simrv}{\,\sim\hspace{-0.05em}}
\DMO{\simlv}{\!\sim\,}
\nc{\simlvi}[1]{\!\sim_{#1}}
%
\newcommand{\nni}{\NNZ \cup \{+\infty\}}
%
% Abschluss von R (7.1.2004):
\newcommand{\rri}{\ol{\RR}} % RR + -+unendlich
%
\newcommand{\ueber}[2]{\genfrac{}{}{0pt}{}{#1}{#2}}% ersetzt atop
%
\newcommand{\floor}[1]{\lfloor #1 \rfloor}
\newcommand{\ceil}[1]{\lceil #1 \rceil}
%
\DMO{\sgn}{sgn} % 16.4.2000
%
\DMO{\rank}{rank} % 1.7.2000
%
\providecommand{\abs}[1]{\lvert #1 \rvert} % 29.8.2000
\providecommand{\norm}[1]{\lVert #1 \rVert} % 29.11.2000
%
\DMO{\Q}{\mf{Q}} % 20.9.2000: die ``qualitive Klasse''
%
% Intervalle (20.12.2000): ``a'' fuer abgeschlossen, ``o'' fuer offen
\providecommand{\iaa}[2]{[#1, #2]}
\providecommand{\ioo}[2]{]#1, #2[}
\providecommand{\iao}[2]{[#1, #2[\,}
\providecommand{\ioa}[2]{\,]#1, #2]}
%
% 13.8.2001: Automorphismen
\DMO{\auto}{Aut}
\DMO{\edom}{End}
%
% 6.9.2001: Menge aller reellen Matrizen
\DMO{\M}{\mc{M}}
\DMO{\Mpm}{\M(\set{-1,0,+1})}
%
% 11.7.2003: Teilbereiche
\newcommand{\tb}[2]{\set{#1, \dots, #2}} % Teilbereich
%
% 22.8.2003: Matrizen und Spur
%
\DMO{\Mat}{\mc{M}}
\DMO{\tr}{tr}
%
% Projektionen
\DMO{\proj}{pr}
%
% Zahlentheorie
\DMO{\teilt}{|}
\nc{\Prim}{\mc{PR}} % Primzahlen
\DMO{\ord}{ord}
\DMO{\ggt}{ggt}
\DMO{\kgv}{kgv}
%
% 1.1.2004 Mengenoperationen
\DMO{\symdif}{\vartriangle} % symmetrische Differenz
\DMO{\addcup}{{\stackrel{\text{\raisebox{-2.2ex}[-0ex][-0ex]{\large$\cdot$}}}{\cup}}}
\DMO{\addbcup}{{\stackrel{\text{\raisebox{-4.2ex}[-0ex][-0ex]{\Large$\cdot$}}}{\bigcup}}}
\DMO{\Rel}{\mf{REL}} % Menge aller binaeren Relationen
\DMO{\Abb}{\mf{MAP}} % Menge der Abbildungen
\DMO{\Tra}{\mf{T}} % Menge der Tranformationen
\DMO{\Per}{\mf{S}} % Menge der Permutationen
\DMO{\Ptr}{\mf{PT}} % partielle Transformationen
\DMO{\cmpl}{\complement}
%
\DMO{\card}{card}
% Realteil und Imaginaerteil (8.1.2004)
\DMO{\re}{Re}
\DMO{\im}{Im}
%
% Auf- und absteigende Fakultaeten (23.4.2004)
\newcommand{\untfak}[2]{(#1)_{\downarrow#2}}
\newcommand{\obfak}[2]{(#1)_{\uparrow#2}}
\DMO{\fak}{fac}
%
% Spezielle Zahlenfolgen (9.10.2005)
\newcommand{\bernoulliz}[1]{b_{#1}}
\newcommand{\bernoullip}[2]{b_{#1}(#2)}
\newcommand{\stirlinge}[2]{s_{#1,#2}}
\newcommand{\stirlingz}[2]{S_{#1,#2}}
\newcommand{\partition}[2]{p_{#1}(#2)}
\DMO{\partitiont}{p}
%
% Ramsey-Zahlen und aehnliches
\DMO{\ramz}{N_R} % Ramsey-Zahl
\DMO{\waez}{N_W} % van der Waerden-Zahl
\DMO{\FvdW}{F_{vdW}} % Klauselmenge, die die van-der-Waerden-Zahl darstellt
\DMO{\FRam}{F_{R}} % Klauselmenge, die Ramsey-Zahl darstellt
\DMO{\arithp}{ap} % Menge aller arithmetischen Progressionen
%
%                                     Graphtheory
%
\DMO{\nachbarn}{\Gamma}
%
% Incidence matrices
%
\DMO{\inzEK}{\mc{I}^{\mr{V}}}
\DMO{\inzEKe}{\mc{I}^{\mr{V}}_1}
\DMO{\inzEKz}{\mc{I}^{\mr{V}}_2}
\nc{\inzEKi}[1]{\mc{I}^{\mr{V}}_{#1}}
\DMO{\inzKE}{\mc{I}^{\mr{E}}}
\DMO{\inzKEe}{\mc{I}^{\mr{E}}_1}
\DMO{\inzKEz}{\mc{I}^{\mr{E}}_2}
\nc{\inzKEi}[1]{\mc{I}^{\mr{E}}_{#1}}
%
% Adjacency matrices
%
\DMO{\adjE}{\mc{A}^{\mr{V}}}
\DMO{\adjEe}{\mc{A}^{\mr{V}}_1}
\DMO{\adjEz}{\mc{A}^{\mr{V}}_2}
\nc{\adjEi}[1]{\mc{A}^{\mr{V}}_{#1}}
\DMO{\adjK}{\mc{A}^{\mr{E}}}
%
% Ranks and degrees
%
\DMO{\degmin}{\ul{\deg}}
\DMO{\degmax}{\ol{\deg}}
\DMO{\degdur}{\widetilde{\deg}}
\DMO{\degmaxl}{\ol{\deg}_{<}}
\DMO{\degl}{\deg_{<}}
%
\DMO{\rankmin}{\ul{\rank}}
\DMO{\rankmax}{\ol{\rank}}
\DMO{\rankdur}{\widetilde{\rank}}
\DMO{\rankmaxl}{\ol{\rank}_{<}}
\DMO{\rankl}{\rank_{<}}
%
% Hypergraphen:
\DMO{\Tr}{Tr} % Transversalen
\DMO{\Ind}{Ind} % independent sets
\DMO{\St}{St} % stars
\DMO{\Ints}{Ints} % intersecting hypergraphs
\DMO{\Cov}{Cov} % coverings
\DMO{\closse}{clo_{\sse}} % Abschluss unter Teilmengenbildung
\DMO{\clospe}{clo_{\supseteq}} % Abschluss unter Obermengenbildung
\DMO{\edgemg}{ML} % Kantenmultigraph
%
% Designs:
\DMO{\PBD}{PBD}
\nc{\BD}[1]{{#1}\text{-}\mr{BD}}
\DMO{\BIBD}{BIBD}
%
% Graphenprodukte (10.10.2004)
%
\DMO{\gpk}{\Box} % Graphenprodukt: kartesisch
\DMO{\gpw}{\times} % Graphenprodukt: schwaches
\DMO{\gps}{\boxtimes} % Graphenprodukt: starkes
\DMO{\gjoin}{\boxdot} % Join von Graphen
\DMO{\gjoinplus}{\boxplus} % Join von Graphen
%
% Ordnungstheorie (26.2.2005)
\DMO{\Ketten}{\mc{L}}
\DMO{\Antiketten}{\mc{A}}
\DMO{\comparable}{\Bumpeq}
\DMO{\incomparable}{\parallel}
%
% Halbringe (30.10.2004)
\DMO{\can}{Can} % kuerzbare Elemente (in Monoiden)
\DMO{\addcan}{Can^+}
\DMO{\multcan}{Can^{\ast}}
\DMO{\sol}{Sol} % loesbare Elemente (in Monoiden)
\DMO{\addsol}{Sol^+}
\DMO{\multsol}{Sol^{\ast}}
\DMO{\inv}{Inv} % invertierbare Elemente (in Monoiden)
\DMO{\addinv}{Inv^+} % Menge der additiv invertierbaren Elemente
\DMO{\multinv}{Inv^{\ast}} % Menge der multiplikativ invertierbaren Elemente
\DMO{\idemp}{Ip} % idempotente Elemente (in Monoiden)
\DMO{\addidemp}{Ip^+}
\DMO{\multidemp}{Ip^{\ast}}
%\DMO{\ker}{ker}
%
% Ganzzahliger Rest und ganzzahlige Division
\DMO{\opmod}{mod}
\DMO{\opdiv}{div}
%
% 21.11.2004
\DMO{\pot}{\PP}
\DMO{\pote}{\PP_f}
\DMO{\potfv}{\overrightarrow{\PP}} % Vorwaerts-Potenzmengenfunktor
\newcommand{\potfvi}[1]{\overrightarrow{\PP}_{\!\!#1}} % Vorwaerts-Potenzmengenfunktor mit Index
\DMO{\potfvn}{\overrightarrow{\PP}^{\!*}} % nichtleerer Vorwaerts-Potenzmengenfunktor
\newcommand{\potfvni}[1]{\overrightarrow{\PP}^{\!*}_{\!\!#1}} % nichtleerer Vorwaerts-Potenzmengenfunktor mit Index
\DMO{\potfr}{\overleftarrow{\PP}} % Rueckwaerts-Potenzmengenfunktor
\newcommand{\potfri}[1]{\overleftarrow{\PP}_{\!\!#1}} % Rueckwaerts-Potenzmengenfunktor mit Index
%
% Topologie
%
\DMO{\nf}{NF} % Nachbarschaftsfilter
\DMO{\erzf}{\mf{GF}} % erzeugter Filter
\DMO{\erzt}{\Ktop} % erzeugte Topologie
\DMO{\abgh}{clo} % abgeschlossene Huelle
\DMO{\offk}{int} % offener Kern
\DMO{\rand}{fr} % alternativ koennte man den Rand auch mit \partial bezeichnen.
\newcommand{\T}[1]{T_{\mr{#1}}}
\DMO{\supp}{supp}
\DMO{\C}{C}
\DMO{\hypo}{\mf{H}} % Hypergraph der nichtleeren offenen Mengen
%
\nc{\sselr}{\sse^{\mapsto}}
\nc{\sserl}{\sse^{\mapsfrom}}
\nc{\spelr}{\spe^{\mapsto}}
\nc{\sperl}{\spe^{\mapsfrom}}
%
% Dimension theory
\nc{\Ccovdim}{\mc{CD}}
\nc{\Cinddim}{\mc{SID}}
\DMO{\inddim}{idim}
\nc{\CInddim}{\mc{LID}}
\DMO{\Inddim}{Idim}
%
\DMO{\sset}{sset} % Nachfolgermenge
%
% ########################################################
% -----------------------------------------------------------------------------------------------------------------
% CATEGORY THEORY
% -----------------------------------------------------------------------------------------------------------------
% ########################################################
\DMO{\obj}{Obj}
\DMO{\mor}{Mor}
\DMO{\mendo}{End}
%
\newcommand{\prodmorf}{\bigotimes} % Produkt einer Morphismenfamilie f_i: X \ra Y_i als Morphismus von X nach dem Produkt der Y_i.
\DMO{\prodmor}{\otimes} % Produkt zweier Morphismen
\newcommand{\coprodmorf}{\bigsqcup} % Koprodukt einer Morphismenfamilie f_i: X_i \ra Y als Morphismus von dem Koprodukt der X_i nach Y.
\DMO{\coprodmor}{\sqcup} %  Koprodukt zweier Morphismen
\newcommand{\mprodmorf}{\prod} % "Mehrfaches" Produkt einer Familie
\DMO{\mprodmor}{\times}
\newcommand{\mcoprodmorf}{\coprod}
\DMO{\mcoprodmor}{\amalg}
%
\newcommand{\Kset}{\mf{SET}} % Mengen
\newcommand{\Kfset}{\Kset_{\mr{f}}} % endliche Mengen
\newcommand{\Kard}{\mc{CARD}} % Kardinalzahlen
\newcommand{\Ord}{\mc{ORD}} % Ordinalzahlen
\newcommand{\Kardm}{\Kard_{\!-1}}
\DMO{\cardsup}{cardsup}
\DMO{\cardsupl}{cardsup_<}
\DMO{\cardmin}{cardmin}
%
\newcommand{\Kkor}{\mf{KOR}} % Korrespondenzen
\newcommand{\liei}{\mr{lu}} % linkseindeutig
\newcommand{\reei}{\mr{ru}} % rechtseindeutig
\newcommand{\lito}{\mr{lt}} % linkstotal
\newcommand{\reto}{\mr{rt}} % rechtsstotal
\newcommand{\kortyp}{\mr{KT}} % Korrespondenztypen
%
\newcommand{\Krel}{\mf{REL}} % Relationale
\newcommand{\Krelk}[2]{\Krel_{#1}\mf{K}_{#2}} % Relationen mit Korrespondenzen
\newcommand{\Krelkr}[2]{\Krelk{#1}{#2}\mf{R}} % Rueckwaertsmorphismen
%
% Quasiorders:
%
\newcommand{\Kqord}{\mf{QORD}} % Quasi-Ordnungen
\newcommand{\Kpord}{\mf{PORD}} % partielle Ordnungen
\newcommand{\Klord}{\mf{LORD}} % lineare Ordnungen
\newcommand{\seb}{\mr{s}} % supremumserhaltend binaer
\newcommand{\seu}{\seb^{\infty}} % supremumserhaltend unendlich
\newcommand{\see}{\seb^{<\infty}} % supremumserhaltend endlich
\newcommand{\seun}{\seb^{\infty}_{\not= \es}} % supremumserhaltend unendlich nichtleer
\newcommand{\seen}{\seb^{<\infty}_{\not= \es}} % supremumserhaltend endlich nichtleer
\newcommand{\ieb}{\mr{i}} % infimumserhaltend binaer
\newcommand{\ieu}{\ieb^{\infty}} % infimumserhaltend unendlich
\newcommand{\iee}{\ieb^{<\infty}} % infimumserhaltend endlich
\newcommand{\ieun}{\ieb^{\infty}_{\not= \es}} % infimumserhaltend unendlich nichtleer
\newcommand{\ieen}{\ieb^{<\infty}_{\not= \es}} % infimumserhaltend endlich nichtleer
\newcommand{\Kvb}{\mf{LT}} % Verb\"ande
\newcommand{\Kuvb}{\ul{\mf{S}}\Kvb} % untere Halbverb\"ande
\newcommand{\Kovb}{\ol{\mf{S}}\Kvb} % obere Halbverb\"ande
\newcommand{\Kvvb}{\mf{C}\Kvb} % vollstaendige Verb\"ande
\newcommand{\Kvuvb}{\mf{C}\Kuvb} % untere vollstaendige Halbverb\"ande
\newcommand{\Kvovb}{\mf{C}\Kovb} % obere vollstaendige Halbverb\"ande
\newcommand{\Kbvb}{\mf{B}\Kvb} % Verb\"ande mit Null und Eins
\newcommand{\Kbuvb}{\mf{B}\Kuvb} % untere Halbverb\"ande mit Null
\newcommand{\Kbovb}{\mf{B}\Kovb} % obere Halbverb\"ande mit Eins
\newcommand{\Kdvb}{\mf{D}\Kvb} % distributive Verb\"ande
\newcommand{\Kkvb}{\mf{CO}\Kvb} % komplementierte Verb\"ande
\newcommand{\Kekvb}{\mf{UCO}\Kvb} % eindeutig komplementierte Verb\"ande
\newcommand{\Kboolvb}{\mf{BO}\Kvb} % boolesche Verb\"ande
\newcommand{\Kvboolvb}{\mf{C}\Kboolvb} % vollstaendige boolesche Verb\"ande
\newcommand{\dist}{\mr{d}} % distributiv
\newcommand{\vdist}{\mr{d}^{\infty}} % vollst\"andig-distributiv
\newcommand{\komp}{\mr{c}} % komplementiert
\newcommand{\ekomp}{\mr{c}_1} % eindeutig komplementiert
\DMO{\qo}{QO} % formation of quasi-order
%
% Gruppoids:
%
\newcommand{\Kgod}{\mf{GOD}} % Gruppoide
\newcommand{\Ksgr}{\mf{SGR}} % Halbgruppen
\newcommand{\Kugod}{\mf{UGOD}} % unitales Gruppoid
\newcommand{\Kmon}{\mf{MON}} % Monoide
\newcommand{\Kcgod}{\mf{CGOD}} % kuerzbare Gruppoide
\newcommand{\Krgod}{\mf{RGOD}} % loesbare Gruppoide
\newcommand{\Kqgp}{\mf{QGP}} % Quasigruppen
\newcommand{\Kcsgr}{\mf{CSGR}} % kuerzbare Halbgruppen
\newcommand{\Kcugod}{\mf{CUGOD}} % kuerzbare unitale Gruppoide
\newcommand{\Krugod}{\mf{RUGOD}} % loesbare unitale Gruppoide
\newcommand{\Kcmon}{\mf{CMON}} % kuerzbare Monoide
\newcommand{\Klop}{\mf{LOP}} % Loops
\newcommand{\Kgr}{\mf{GRP}} % Gruppen
\newcommand{\Kagod}{\mf{AGOD}} % abelsche Gruppoide
\newcommand{\Kasgr}{\mf{ASGR}} % abelsche Halbgruppen
\newcommand{\Kaugod}{\mf{AUGOD}} % abelsche unitales Gruppoid
\newcommand{\Kamon}{\mf{AMON}} % abelsche Monoide
\newcommand{\Kacgod}{\mf{ACGOD}} % abelsche kuerzbare Gruppoide
\newcommand{\Kargod}{\mf{ARGOD}} % abelsche loesbare Gruppoide
\newcommand{\Kaqgp}{\mf{AQGP}} % abelsche Quasigruppen
\newcommand{\Kacsgr}{\mf{ACSGR}} % abelsche kuerzbare Halbgruppen
\newcommand{\Kacugod}{\mf{ACUGOD}} % abelsche kuerzbare unitale Gruppoide
\newcommand{\Karugod}{\mf{ARUGOD}} % abelsche loesbare unitale Gruppoide
\newcommand{\Kacmon}{\mf{ACMON}} % abelsche kuerzbare Monoide
\newcommand{\Kalop}{\mf{ALOP}} % abelsche Loops
\newcommand{\Kagr}{\mf{AGRP}} % abelsche Gruppen
%
\newcommand{\Klact}[2]{{}_{#1} #2} % Links-Aktionen (oder -Operation)
\newcommand{\Kract}[2]{#2_{#1}} % Rechts-Aktionen (oder -Operation)
\newcommand{\Kact}{\mf{ACT}} % Aktionen
\newcommand{\Kopt}{\mf{OPT}} % Operationen
%
% Rings:
%
\newcommand{\Kring}{\mf{RNG}} % Ringe
\newcommand{\Kkring}{\mf{CRNG}} % kommutative Ringe
\newcommand{\Khring}{\mf{SRNG}} % Halbringe
\newcommand{\Kpring}{\mf{PRNG}} % Pseudoringe
\newcommand{\Kphring}{\mf{PSRNG}} % Pseudohalbringe
\newcommand{\Kkhring}{\mf{CSRNG}} % kommutative Halbringe
\newcommand{\Kkpring}{\mf{CPRNG}} % kommutative Pseudoringe
\newcommand{\Kkphring}{\mf{CPSRNG}} % kommutative Pseudohalbringe
\newcommand{\Kprering}{\mf{PRG}} % Pr\"aeringe
%
\newcommand{\Kkat}{\mf{KAT}} % Kategorie der Kategorien
\newcommand{\Kpaddkat}{\mf{PAKAT}} % praeadditive Kategorien
\newcommand{\Knpaddkat}{\mf{NPAKAT}} % normale praeadditive Kategorien
\newcommand{\Ksaddkat}{\mf{SAKAT}} % semiadditive Kategorien
\newcommand{\Kaddkat}{\mf{AKAT}} % additive Kategorien
\newcommand{\Kkatn}{\mf{KAT}_2} % Kategorie der Kategorien mit natuerlichen Transformationen als Morphismen
%
\DMO{\Kfun}{\mf{FUN}} % Funktorkategorie
\DMO{\nattr}{NAT} % natuerliche Transformationen
%
% Topology:
%
\newcommand{\Ktop}{\mf{TOP}} % Kategorie der topologischen R\"aume
\newcommand{\modhom}{\mf{H}} % modulo Homotopie
\newcommand{\Ktophom}{\Ktop\modhom} % Kategorie der topologischen R\"aume modulo Homotopie
\newcommand{\Kptophom}{\mf{TOP}_*\modhom} % Kategorie der punktierten topologischen R\"aume modulo Homotopie
\newcommand{\Ktoptmhom}{\Ktop_{[1]}\modhom_1} % Topologische Raeume mit Teilmengen modulo Homotopie
\newcommand{\Katop}{\mf{ATOP}} % Kategorie der A-topologischen Raeume
%
\DMO{\Kdelta}{\Delta} % Mengen {0, ..., n-1} mit monoton steigenden Abbildungen
\DMO{\Kdeltas}{\Delta_{\mr{s}}} % Mengen {0, ..., n-1} mit streng monoton steigenden Abbildungen
\DMO{\Kdeltat}{\Delta^t}
\DMO{\Kdeltast}{\Delta_{\mr{s}}^t}
\DMO{\Kdeltap}{\Delta^+}
\DMO{\Kdeltasp}{\Delta_{\mr{s}}^+}
\DMO{\Kdeltapt}{{\Delta^+}^t}
\DMO{\Kdeltaspt}{{\Delta_{\mr{s}}^+}^t}
%
\DMO{\symm}{Symm} % Symmetrisierung einer Kategorie
%
% Graphs:
%
\newcommand{\Kdgg}{\mf{DGG}} % directed general graphs
\newcommand{\Kdg}{\mf{DG}} % directed graphs
\newcommand{\Kdgl}{\mf{DGL}} % directed graphs with loops
\newcommand{\Kgg}{\mf{GG}} % general graphs
\newcommand{\Kg}{\mf{GR}} % graphs
\newcommand{\Kgl}{\mf{GRL}} % graphs with loops
\newcommand{\Khg}{\mf{HGR}} % hypergraphs
\newcommand{\Khgi}{\Khg^{\infty}} % hypergraphs with infinite edges
\newcommand{\Kfhg}{\Khg_{\mr{f}}} % finite hypergraphs 
\newcommand{\Kghg}{\mf{G}\Khg} % general hypergraphs
\newcommand{\Kghgi}{\Kghg^{\infty}} % general hypergraphs with infinite edges
\newcommand{\Kfghg}{\Kghg_{\mr{f}}} % finite general hypergraphs
\newcommand{\Kbip}{\mf{BIP}} % bipartite Systeme
%
\DMO{\dgg}{DGG} % formation of directed general graphs
\DMO{\dgl}{DGL} % formation of directed graphs with loops
\DMO{\dg}{DG} % formation of directed graphs
\DMO{\ugg}{UGG} % zugrundeliegender allgemeiner Graph
\DMO{\ugl}{UGL} % zugrundeliegender Graph mit Schlingen
\DMO{\ug}{UG} % zugrundeliegender Graph
\DMO{\sg}{SG} % Symmetrisierung eines Graphen
\DMO{\sgl}{SG} % Symmetrisierung eines Graphen mit Schlingen
%
\DMO{\bip}{bip} % Das bipartite System assoziiert mit einem allgemeinen Hypergraphen
\DMO{\hyp}{hyp} % der allgemeine Hypergraph assoziiert mit einem bipartiten System.
%
\DMO{\cat}{CAT} % formation of categories (e.g., the free category created by a graph)
%
% Base sets with set systems
%
\newcommand{\Ksetsys}{\mf{SSYS}}
% ########################################################
% ----------------------------------------------------------------------------------------------------------------
% SAT NOTATIONS
% -----------------------------------------------------------------------------------------------------------------
% ########################################################
\newcommand{\Va}{\mc{V\hspace{-0.4mm}A}}
\newcommand{\Lit}{\mc{LIT}}
\newcommand{\Cl}{\mc{CL}}
\newcommand{\Cls}{\mc{CLS}}
\newcommand{\Tcls}{3\mbox{--}\Cls}
\newcommand{\Pcls}[1]{#1\mbox{--}\Cls}
\newcommand{\Clsab}[2]{\Cls\mbox{-}(#1, #2)}
\newcommand{\Clsr}[1]{\Cls\mbox{-}#1}
\newcommand{\Pass}{\mc{P\hspace{-0.32em}ASS}}
\newcommand{\Tass}{\mc{T\hspace{-0.35em}ASS}}
\newcommand{\Sat}{\mc{SAT}}
\newcommand{\Tsat}{3\mbox{-}\Sat}
\newcommand{\Usat}{\mc{USAT}}
\newcommand{\Usati}[1]{\Usat_{\!\!#1}}
\newcommand{\Musat}{\mc{M\hspace{0.8pt}USAT}}
\newcommand{\Musati}[1]{\Musat_{\!\!#1}}
\newcommand{\Smusat}{\mc{S}\Musat}
\newcommand{\Smusati}[1]{\Smusat_{\!\!#1}}
\newcommand{\Mmusat}{\mc{M}\Musat}% marginale minimal unerf\"ullbare Klms (25.4.2001)
\newcommand{\Mmusati}[1]{\Mmusat_{\!\!#1}}
\newcommand{\Fpass}{{\mc{F}}\Pass}
%
\newcommand{\AF}{\mc{A}}% ausgezeichnete KNF
\newcommand{\Clash}{\mc{HIT}} % schwach-resolvierbare Klauselmengen (global umdefiniert 24.7.2003)
\newcommand{\Clashi}[1]{\Clash_{\!\!#1}}
\newcommand{\Uclash}{\mc{U}\Clash} % unerfuellbare Treffklauselmengen
\newcommand{\Uclashi}[1]{\Uclash_{\!\!#1}}
\newcommand{\Sclash}{\mc{U}\Clash} % stark-resolvierbare Klauselmengen (global umdefiniert 24.7.2003)
\newcommand{\Sclashk}[1]{\Sclash_{\! #1}}
\newcommand{\Mclash}{\mc{M}\Clash} % Multi-hitting 25.9.2004
\newcommand{\Pclash}[1]{{#1}_{\nless}\mbox{\!--}\Clash} % Klausell\"ange exakt k 13.11.2005; einen allgemeinen Standard einfuehren fuer solche exakten Klausellaengenangaben
\newcommand{\Puclash}[1]{{#1}_{\nless}\mbox{--}\Uclash}
\DMO{\maxnpuclash}{NH}
\DMO{\munpuclash}{\mu{}NH}
\DMO{\hdef}{\delta_{\mr{h}}} % hermitean defect 6.1.2004
\DMO{\rdef}{\delta_{\mr{r}}} % rank defect 12.9.2004
\newcommand{\Mclshd}{\Mcls_{\hdef}} % 24.7.2003
\newcommand{\Clshd}{\Cls_{\hdef}} % 24.1.2004
\newcommand{\Mmhd}{\Mclshd(1)} % 24.7.2003
\newcommand{\Mclsp}{\Mcls_{i_+}} % 24.7.2003
\newcommand{\Mclsn}{\Mcls_{i_-}} % 24.7.2003
\newcommand{\Mclsh}{\Mcls_h} % 24.7.2003
\newcommand{\Clsp}{\Cls_{i_+}} % 24.1.2004
\newcommand{\Clsn}{\Cls_{i_-}} % 24.1.2004
\newcommand{\Clsh}{\Cls_h} % 24.1.2004
\newcommand{\Musath}{\Musat_{\!h}} % 24.7.2003
\newcommand{\Musatd}{\Musat_{\!\delta}} % 24.7.2003
\newcommand{\Smusatd}{\Smusat_{\!\delta}} % 24.7.2003
\newcommand{\Mclsds}{\Mcls_{\delta^*}} % 24.7.2003
\newcommand{\Trcls}{\mc{TCLS}}
\newcommand{\Gtrcls}{\mc{GTCLS}}
\newcommand{\Satein}{\Sat\mbox{-}1}
\newcommand{\Lsat}{\mc{L}\Sat}
\newcommand{\Ho}{\mc{HO}}
\newcommand{\Rho}{\mc{R}\Ho}
\newcommand{\Qho}{\mc{Q}\Ho}
\newcommand{\Lean}{\mc{LEAN}}
\newcommand{\Llean}{\mc{L}\Lean}
\newcommand{\Umusat}{\mc{U}\Musat}
\newcommand{\Mlean}{\mc{M}\Lean}
\newcommand{\Mleani}[1]{\Mlean_{\!#1}}
\newcommand{\Msat}{\mc{M}\Sat}
\newcommand{\Lsatz}{{\Lsat\!}_0}
\newcommand{\Satv}[1]{\Sat\!_{#1}} % 15.8.2001
\newcommand{\Psat}{\mc{P}\Sat}
\newcommand{\Plean}{\mc{P}\Lean}
\newcommand{\Mcls}{\mc{M}\Cls} % 27.12.2002
\newcommand{\Pmcls}[1]{#1\mbox{--}\Mcls}
\newcommand{\Scls}{\mc{SCL}} % 27.5.2004
\newcommand{\Ssat}{\Sat} % 9.6.2004
\newcommand{\Dcls}{\mc{D}\Cls} % 28.11.2004 (diagonale Klauselmengen)
%
\DMO{\nulli}{null} % 17.5.2004
%
\newcommand{\dnf}{\mathrm{DNF}}
\newcommand{\pdnf}{p\mbox{-}\mathrm{DNF}}
\newcommand{\tdnf}{3\mbox{-}\mathrm{DNF}}
\newcommand{\twdnf}{2\mbox{-}\mathrm{DNF}}
\newcommand{\dnfa}{\mathrm{DNF}\mbox{-}(1, s)}
\newcommand{\dnfoi}{\mathrm{DNF}\mbox{-}(1, \infty)}
\newcommand{\dnfot}{\mathrm{DNF}\mbox{-}(1, 2)}
\newcommand{\tdnfot}{3\mbox{-}\mathrm{DNF}\mbox{-}(1, 2)}
\newcommand{\pdnfa}{p\mbox{-}\mathrm{DNF}\mbox{-}(1, s)}
%
\DMO{\res}{\diamond}
%
\newcommand{\pcl}[1]{${\SST\le} #1$-clause}
\newcommand{\pkl}[1]{${\SST\le} #1$-Klausel}
%
\DMO{\lit}{lit}
\DMO{\var}{var}
\DMO{\val}{val}
%
\DMO{\dpl}{DP}
\newcommand{\dpi}[1]{\dpl_{\!#1}}
%
\newcommand{\sateq}{\overset{\mr{sat}}{\equiv}}
%
\DMO{\comp}{Comp}
\DMO{\compex}{\comp_{ER}}
\DMO{\compr}{\comp_R}
\DMO{\comptr}{\comp_{tR}}
\newcommand{\Us}{\mc{U}} % August 2000
\DMO{\comptru}{\comp_{tR(\Us)}}
\DMO{\compru}{\comp_{R(\Us)}}
\providecommand{\comptruv}[1]{\comp_{\mr{tR}(#1)}}
\providecommand{\compruv}[1]{\comp_{\mr{R}(#1)}}
\newcommand{\Sa}{\mc{S}} % September 2001
%
\newcommand{\php}{\mathrm{PHP}}
\DMO{\pebf}{PF} % pebbling formulas (veraendert am 18.5.2001; vormals \pf)
%
\DMO{\rt}{root}
\DMO{\nds}{nds}
\DMO{\lvs}{lvs}
\DMO{\cls}{cls}
\DMO{\nlvs}{\#lvs}
\DMO{\newcommandls}{\#cls}
\DMO{\ds}{ds}
\DMO{\dst}{ds_T}
\DMO{\dsg}{ds_G}
\DMO{\dpr}{dp}
\DMO{\dprt}{dp_T}
\DMO{\dprg}{dp_G}
\DMO{\ind}{in}
\DMO{\indg}{in_G}
\DMO{\outd}{out}
\DMO{\outdg}{out_G}
\DMO{\height}{ht}
%
\DMO{\peb}{peb} % 20.2.2000
%
\DMO{\taum}{\tau_{max}}
%
\newcommand{\pab}[1]{\langle #1 \rangle}
\newcommand{\pao}[2]{\langle #1 \ra #2 \rangle}
\newcommand{\pat}[4]{\langle #1 \ra #2, #3 \ra #4 \rangle}
%
\newcommand{\Bt}{\mc BT}
%
\newcommand{\Cnf}{\mathrm{CNF}}
\DMO{\pc}{pc}
%
\DMO{\aut}{Auk}
\DMO{\laut}{LAuk}
\DMO{\maut}{MAuk}
\newcommand{\A}{\mc{A}} % Autarkie-System
\DMO{\nv}{N}
\DMO{\na}{\nv_a}
\DMO{\nA}{\nv_{\A}} % Normalform
\DMO{\nla}{\nv_{la}}
\DMO{\nbla}{\nv_{bla}}
\DMO{\nma}{\nv_{ma}}
\DMO{\npa}{\nv_{pa}}
\DMO{\baut}{BAuk} % balancierte Autarkien
\DMO{\blaut}{BLAuk} % balancierte lineare Autarkien
\DMO{\paut}{PAut} % pure Autarkien
\newcommand{\SatA}{\Sat\!_{\A}}
\newcommand{\LeanA}{\Lean\!_{\A}}
%
% relativierte Resolution (21.2.20000)
\DMO{\resouz}{\overset{\Us, 0}{\vdash}}
\DMO{\resouo}{\overset{\Us, 1}{\vdash}}
\DMO{\resouk}{\overset{\Us,\, k}{\vdash}}
\DMO{\resou}{\,\overset{\Us}{\vdash}\,}
\DMO{\resour}{\,\overset{\Us_0}{\vdash}\,}
\DMO{\resourz}{\,\overset{\Us_0, 0}{\vdash}\,}
\DMO{\uresouk}{\resouk\hspace{-0.6em}\mbox{\raisebox{0.8ex}{\tiny u}}}
\DMO{\bresouk}{\resouk\hspace{-0.6em}\mbox{\raisebox{0.8ex}{\tiny b}}}
\DMO{\iresouk}{\resouk\hspace{-0.6em}\mbox{\raisebox{0.8ex}{\tiny i}}}
\DMO{\resok}{\overset{k}{\vdash}} % 27.8.2000
\newcommand{\resokv}[1]{\overset{#1}{\vdash}}
\newcommand{\resoukv}[1]{\overset{\Us, \,#1}{\vdash}}
%
\DMO{\wid}{wid} % width
\DMO{\widl}{\hspace*{-1.5pt}wid}
\DMO{\widb}{\sideset{^{\mr{b}}}{}\widl}
\DMO{\widi}{\sideset{^{\mr{i}}}{}\widl}
\DMO{\cwid}{\mc{W}} % classes
\DMO{\cwidl}{\hspace*{-1pt}\mc{W}} % classes
\DMO{\cwidb}{\sideset{^{\mr{b}}}{}\cwidl}
\DMO{\cwidi}{\sideset{^{\mr{i}}}{}\cwidl}
%
\DMO{\modp}{mod_p}
\DMO{\modt}{mod_t}
\DMO{\moda}{\mf{S}} % 15.1.2003; ``allgemeine Modelle''
\DMO{\modf}{\mf{F}} % 27.11.2004; ``falsifizierende Modelle''
%
% 31.12.2000: Systems of Problem Instances
\newcommand{\PI}{\mc{PI}}
\newcommand{\Spi}{\mc{S}\PI}
\newcommand{\Upi}{\,\mc{U}\PI}
%
% 14.7.2005: Hypergraphs of minimally unsatisfiable sub-clause-sets etc.
\DMO{\mus}{MU}
\DMO{\mss}{MS}
\DMO{\cmus}{CMU}
\DMO{\cmss}{CMS}
%
% Konfliktmatrizen
%
\DMO{\scf}{CM} % symmetric conflict matrix
\DMO{\acf}{DCM} % asymmetric conflict matrix
\DMO{\cmg}{cmg} % conflict multigraph
\DMO{\cmdg}{cmdg} % conflict multidigraph
\DMO{\cg}{cg} % conflict graph
\DMO{\rsg}{rsg} % resolution graph
\DMO{\acg}{acg} % accordance graph
\DMO{\acmg}{acmg} % accordance graph
\DMO{\nscf}{bcp} % symmetric conflict number
\DMO{\nacf}{bcp_d} % asymmetric conflict number
\DMO{\bcp}{bcp} % biclique partition number
%
%
% Minimale Klauselnlaenge ... (aus meiner SAT-Vorlesung (CS-342) in 2003/04)
%
\DMO{\nsat}{\#sat}
\DMO{\maxsat}{maxsat}
%
\DMO{\pmin}{\ul{rk}}
\DMO{\pmax}{rk}
\DMO{\pav}{\widetilde{rk}}
%
\DMO{\ldeg}{ldg}
\DMO{\minldeg}{\ul{ldg}}
%
\DMO{\vdeg}{vdg}
\DMO{\minvdeg}{\ul{vdg}}
\DMO{\avvdeg}{\widetilde{vdg}}
%
\DMO{\cldeg}{cldg} % complementary literal degree
%
% Spezielle Klauselmengen
\DMO{\Inj}{Inj}
%
% 13.6.2004
\newcommand{\OKsolver}{\texttt{OKsolver}}
\newcommand{\OKlibrary}{\texttt{OKlibrary}}
\newcommand{\OKgenerator}{\texttt{OKgenerator}}
\newcommand{\OKplatform}{\texttt{OKplatform}}
\newcommand{\OKdatabase}{\texttt{OKdatabase}}
\newcommand{\OKsystem}{\texttt{OKsystem}}
%
% ########################################################
% ----------------------------------------------------------------------------------------------------------------
% COMPLEXITY THEORY
% -----------------------------------------------------------------------------------------------------------------
% ########################################################
%
\DMO{\timem}{time}
\DMO{\spacem}{space}
\DMO{\dtime}{DTime}
\DMO{\dspace}{DSpace}
\nc{\Con}{\mr{Con}}
\nc{\Lin}{\mr{Lin}}
\nc{\Pol}{\mr{Pol}}
\nc{\ExL}{\mr{ExL}}
\nc{\ExP}{\mr{ExP}}
\nc{\CTime}{\mr{CTime}}
\nc{\CSpace}{\mr{CSpace}}
\nc{\LTime}{\mr{LTime}}
\nc{\LSpace}{\mr{LSpace}}
\nc{\PTime}{\mr{PTime}}
\nc{\PSpace}{\mr{PSpace}}
\nc{\ELTime}{\mr{ELTime}}
\nc{\ELSpace}{\mr{ELSpace}}
\nc{\EPTime}{\mr{EPTime}}
\nc{\EPSpace}{\mr{EPSpace}}
\nc{\Np}{\mr{NP}}
\nc{\Conp}{\text{co-NP}}
\nc{\Poly}{\mr{P}}
%
%%% Local Variables:
%%% mode: latex
%%% TeX-parse-self: t
%%% TeX-auto-save: t
%%% TeX-master: "Definitionen"
%%% End:
